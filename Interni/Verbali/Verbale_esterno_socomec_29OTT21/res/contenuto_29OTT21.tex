\section{Ordine del giorno}
\begin{itemize}
    \item Quale supporto viene garantito
    \item Cosa viene fornito
    \item Presenza di una fase di apprendimento guidato
    \item Limiti sui linguaggi di programmazione utilizzabili
    \item Modalità di assistenza Tecnico-Cliente 
    \item Preferenze di dispositivo/Sistema Operativo
    \item Visibilità delle schermate relative alle
    statistiche anche da PC
    \item Necessità di scendere nei "tecnicismi"
    \item Significato “oggetto di
    verifica di fattibilità in corso
    d’opera” della parte opzionale
    della Virtual Display
    \item Opinione dell'azienda riguardo la parte più
    critica del progetto  
\end{itemize}
\section{Contenuto}
Durata della conferenza: 45 minuti
\newline
\begin{longtable}{|p{7cm}|p{8cm}|}
    \hline
    & \\
    Questione & Conclusione\\
    & \\
    \hline
    & \\

    % Questione 1
    Quale supporto viene garantito    
    &
    L'azienda mette a disposizione
    una squadra di esperti,
    disposti a fornire supporto
    tecnico durante l'intero
    svolgimento del progetto.\\

    & \\

    % Questione 2
    Cosa viene fornito
    &
    Verrà messo a disposizione del
    materiale di apprendimento
    circa l'UPS, scalette per il
    supporto allo sviluppo di
    alcune parti del sistema e
    simulatori per la verifica.\\

    & \\

    % Questione 3
    Presenza di una fase di apprendimento guidato  
    &
    L'azienda (su richiesta) si
    mette a disposizione anche
    per una fase di apprendimento
    guidato.\\

    & \\

    % Questione 4
    Limiti sui linguaggi di programmazione utilizzabili     
    &
    Non vengono imposti limiti sui
    linguaggi e sulle tecnologie da
    usare: se queste dovessero
    risultare a pagamento le spese
    saranno a carico del proponente
    (previa discussione e approvazione).\\

    & \\

    % Questione 5
    Modalità di assistenza Tecnico-Cliente  
    &
    Durante l'assistenza cliente-
    tecnico si preferisce una
    modalità di chiamata, per poi
    eventualmente sviluppare
    anche quella video-chiamata;\\
    & \\
    \hline
    & \\
    &
    la modalità "chat" non è
    indispensabile.\\
    & \\

    % Questione 6
    Preferenze di dispositivo/Sistema Operativo
    &
    Dal lato cliente dovrà essere
    sviluppata una applicazione
    mobile, multipiattaforma o
    nativa, in italiano e
    graficamente curata, mentre
    dal lato tecnico solo desktop,
    anche dall'aspetto più
    "spartano" ed essenziale,
    finalizzato alla visualizzazione
    dei dati ricevuti durante la
    richiesta di supporto.\\

    & \\

    % Questione 7
    Visibilità delle schermate relative alle
    statistiche anche da PC     
    &
    Solo operazioni di
    visualizzazione delle
    informazioni dell’UPS sia per
    cliente che tecnico. Il tecnico
    non si collega all’UPS ma
    riceve dal cliente
    (automaticamente e una sola
    volta all’inizio della
    comunicazione cliente-
    tecnico) un pacchetto testuale
    che riassume tutte le
    informazioni dell’UPS
    sotto forma di codici. Sul lato
    tecnico abbiamo massima
    libertà in modo che sia quanto
    più semplice possibile
    svilupparlo.\\

    & \\

    % Questione 8
    Necessità di scendere nei "tecnicismi"
    &
    Nessuna conoscenza iniziale
    necessaria (anche grazie agli
    schemi da loro forniti).\\
    
    & \\
    
    % Questione 9
    Significato “oggetto di
    verifica di fattibilità in corso
    d’opera” della parte opzionale
    della Virtual Display
    &
    Per l'uso della tecnologia
    Bluetooth si richiede di dotarsi
    di veri UPS, pertanto il
    requisito sarà valutato in corso
    d'opera discutendone con il
    Professore di riferimento.\\

    & \\

    % Questione 10
    Opinione dell'azienda riguardo la parte più
    critica del progetto     
    &
    Viene consigliato di fare una
    buona analisi dei requisiti, per
    poi passare alla progettazione
    sviluppando l'app seguendo
    dei design pattern. Si richiede
    inoltre di porre particolare
    attenzione al requisito delle
    chiamate.\\

    & \\
    \hline
\end{longtable}

\section{Considerazioni Finali}

\paragraph{}
A seguito dell'incontro si é generato particolare interesse verso il progetto
proposto, con nota di merito verso l'azienda che ha saputo fornire i dettagli richiesti
in modo chiaro e convincente.
\newline
Il team prende in considerazione di aggiudicarsi il capitolato corrispondente.