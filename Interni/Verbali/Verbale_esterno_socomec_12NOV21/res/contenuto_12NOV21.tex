\section{Ordine del giorno}
\begin{itemize}
    \item Modalità di condivisione della Repository
    \item Fine degli emulatori e come dovranno essere usati
    \item Valutazione della possibilità di utilizzare il
    linguaggio React Native per uno sviluppo
    multipiattaforma, a seguito dell'analisi dei requisiti
    \item Tecnologie da loro internamente utilizzate
    \item Quando e come sarà possibile
    accedere alla documentazione di supporto
    \item Modalità di contatto predilette
\end{itemize}
\section{Contenuto}
Durata della conferenza: 60 minuti
\newline
\begin{longtable}{|p{7cm}|p{8cm}|}
    \hline
    & \\
    Questione & Conclusione\\
    & \\
    \hline
    & \\

    % Questione 1
    Modalità di condivisione della Repository     
    &
    La repository dovrà essere
    resa a loro pubblica, con
    accesso read, per scopo
    consultivo. L'accesso in
    modalità write sarà
    eventualmente concesso solo
    in casi di particolare necessità.\\

    & \\

    % Questione 2
    Fine degli emulatori e
    come dovranno essere usati   
    &
    Verrà proposto un emulatore
    proprietario che gira su
    macchine Windows. Da un
    socket verrano estrapolati i
    dati secondo protocollo
    Modbus; e l'indirizzo per
    generare la risposta sarà a
    partire da un file xml. Dovremo
    noi creare il contorno di questo
    algoritmo, scritto in linguaggio
    C, con il core di quello che fa
    l’emulatore.\\

    & \\

    % Questione 3 
    Valutazione della possibilità di utilizzare il
    linguaggio React Native per uno sviluppo
    multipiattaforma, a seguito dell'analisi dei requisiti   
    &
    Considerata interessante la
    proposta, sono disponibili a
    valutare l'utilizzo di tale
    linguaggio considerando, ovviamente, il
    complessivo da fare.
    Apprezzate in generale le
    proposte di nuove idee, anche
    da approfondire assieme.\\

    & \\

    % Questione 4
    Tecnologie da loro internamente utilizzate   
    &
    Utilizzano linguaggi come il C
    per lavorare sui micro-
    controllori (assieme ad
    assembly); C++ per esempio
    negli aspetti grafici; script in
    python; visual basic per
    scripting all’interno di file
    Excel. Per la parte di Sistemi
    Operativi ad alto livello invece
    utilizzano .NET framework (per
    prodotti di manutenzione) e
    ancora vari linguaggi di
    scripting.\\
    & \\
    \hline
    & \\
    &
    Sono soliti
    sviluppare su workstation,
    server, software di
    integrazione a data
    center ecc.\\
    
    & \\

    % Questione 5
    Quando e come sarà possibile
    accedere alla documentazione
    di supporto     
    &
    La documentazione potrà già
    essere disponibile, su nostra
    richiesta, a seguito
    dell'assegnazione dell'appalto.\\

    & \\

    % Questione 6
    Modalità di contatto predilette    
    &
    Per un primo momento la
    modalità principale di contatto
    sarà tramite email, per poi
    eventualmente trovare
    ulteriori strumenti se
    necessario. Si dichiarano
    sempre disponibili per ogni
    evenienza più o meno urgente.\\

    & \\
    \hline
\end{longtable}

\section{Considerazioni Finali}

\paragraph{}
Le buone impressioni verso l'azienda si consolidano, aumentando l'interesse verso
il progetto.
L'incontro è risultato utile per approfondire argomenti più dettagliati circa le
modalità di lavoro.\\
Il Team decide definitivamente di scegliere il progetto del proponente interessato.