\section{Ordine del giorno}
\begin{itemize}
    \item Resoconto secondo incontro con il proponente Socomec
    \item Metodologie di tipo agile
    \item Strumenti di collaborazione
    \item Norme
\end{itemize}
\section{Contenuto}
Durata della conferenza: 75 minuti
\newline
\begin{longtable}{|p{7cm}|p{8cm}|}
    \hline
    & \\
    Questione & Conclusione\\
    & \\
    \hline
    & \\

    % Questione 1
    Resoconto secondo incontro con il proponente Socomec
    &
    Il team ha proposto un breve resoconto sull'incontro
    avvenuto con il proponente Socomec, stilando ulteriori
    punti a favore sulla scelta del capitolato.\\

    & \\

    % Questione 2
    Metodologie di tipo agile
    &
    Si è discusso di come organizzare il lavoro al fine di
    evitare sprechi di tempo e rimanere sempre aggiornati.
    A seguito di varie proposte si è deciso di adottare la
    metodologia SCRUM con sprint di 2 settimane, facendo
    combaciare il termine di uno sprint al cambio dei ruoli.\\

    & \\

    % Questione 3
    Strumenti di collaborazione
    &
    Si sono prese in considerazione varie proposte da parte
    dei membri del team per decidere quali software
    utilizzare in questa prima fase di candidatura.
    Si è giunti a definire la seguente lista con associati
    programmi e scopo:\\ &
    - GitHub: Repo doc / SCRUM Board \\ &
    - Google Drive: Archiviazione bozze, file sperimentali \\ &
    - Microsoft Excel, PowerPoint, LaTeX: Redazione documentazione e slideshow\\ &
    - Discord: Daily SCRUM, comunicazione collaborativa \\ &
    - Microsoft Teams, GMail: comunicazione con SOCOMEC \\ &
    - Telegram: chat testuale miscellanea\\ &
    - WhatsApp: annunci importanti \\ &
    (problemi di notifiche con Telegram)\\

    & \\
    & \\
    & \\
    \hline
    & \\

    % Questione 4
    Norme
    &
    Si è iniziato a stilare un documento con le norme da
    seguire nei casi fin'ora affrontati, come la nomenclatura
    dei verbali. Il documento è disponibile nella repo
    docs/interni.\\
    
    & \\

    \hline
\end{longtable}

\section{Considerazioni Finali}

\paragraph{}
Incontro utile per consolidare la scelta finale e
scegliere gli strumenti collaborativi più adatti.
Le norme di progetto richiederaano un altro incontro.