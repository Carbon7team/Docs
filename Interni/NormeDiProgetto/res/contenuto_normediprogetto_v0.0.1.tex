\begin{center}
    \textbf{Ciclo di vita di un documento e versionamento}
\end{center}

\section{Documenti}
Ogni documento è successione di date fasi di vita, quali:

\begin{enumerate}
    \item \textbf{Pianificazione} \\
    Si identificano le necessità del documento, per delinearne il contenuto fondamentale.
    \item \textbf{Redazione} \\
    Viene costruito lo scheletro del documento e messo per esteso da un redattore/autore. La prima stesura cerca di coprire tutte le necessità richieste, fornendo un contenuto quanto più completo.
    \item \textbf{Revisione} \\
    Il documento previa Redazione, viene revisionato sul lato grammaticale e di contenuto. In caso di errori il documento tornerà alla fase precedente per essere così aggiornato e corretto.
    \item \textbf{Approvazione} \\
    Il documento viene assunto come revisionato e pertanto corretto in forma e contenuto; viene dunque pubblicato e reso disponibile.\\
\end{enumerate}

Ogni documento dovrà essere essere soggetto a versionamento, nella forma $x.y.z$.\\

Il versionamento è definito come segue:
\begin{itemize}
    \item Numero $z$\\
    Tale numero verrà aggiornato ogni qual volta verrà effettuata una stesura del documento.
    \item Numero $y$\\
    Tale numero verrà aggiornato ogni qual volta verrà effettuata una revisione del documento.
    \item Numero $x$\\
    Tale numero verrà aggiornato ogni qual volta verrà effettuata una approvazione del documento.\\
\end{itemize}
 
La versione di ogni documento partirà, pertanto, dalla $0.0.1$.\\

Ogni documento sarà quindi composto da un nome seguito dalla sua attuale versione.\\
Ogni documento dovrà avere la sua cartella, con i file in \LaTeX e relativo contenuto.
La cartella contenente i file dei singoli documenti dovrà essere strutturata come segue:\
\begin{itemize}
    \item Nome cartella: <nomeDocumento> <versione>\\
    \item Contenuto cartella:\\
    \begin{itemize}
        \item file main.tex
        \item cartella res, cui contenuto è composto da:
        \begin{itemize}
            \item cartella con il logo del Team
            \item copertina nella forma copertina <nomeDocumento>.tex
            \item file di contenuto nella forma <contenuto> <infoaggiuntiva> <versione>.tex
        \end{itemize}
    \end{itemize}
\end{itemize}

La copertina di ogni documento dovrà avere i seguenti elementi:\\
\begin{enumerate}
    \item Destinazione del documento: Interno o Esterno
    \item Logo del Carbon7team
    \item Info di contatto del Team
    \item Data ultimo versionamento
    \item Titolo del documento
    \item Elenco dei redattori/autori
    \item Elenco dei revisori
    \item Sommario
\end{enumerate}

Ad eccezione del documento di Candidatura, dove a seguito del Titolo del documento vi saranno i componenti del gruppo e un sommario.

\subsection{Verbali}
I verbali, sia in forma interna che esterna, non sono soggetti al versionamento, in quanto non soggetti a modifiche nel tempo.\\ 
Quest'ultimi dovranno sempre essere caricati sia come documento in \LaTeX sia come PDF nella Repository, al percorso Docs/Interni/Verbali.
La cartella contenente i file relativi ad ogni singolo verbale dovrà essere strutturata come segue:\\
\begin{itemize}
    \item Nome cartella: verbale  <interno/esterno>  <azienda/gruppo>  <data>\\
    \item Contenuto cartella:\\
    \begin{itemize}
        \item file: main.tex
        \item cartella res, cui contenuto è composto da:
        \begin{itemize}
            \item cartella con il logo del Team
            \item copertina nella forma copertina <data>.tex
            \item file di contenuto nella forma contenuto <infoaggiuntiva> <data>.tex
        \end{itemize}
    \end{itemize}
\end{itemize}

La copertina di ogni verbale dovrà avere i seguenti elementi:\\
\begin{enumerate}
    \item Intestazione Verbale
    \item Logo del Carbon7team
    \item Info di contatto del Team
    \item Data a cui fa riferimento il verbale
    \item Titolo del documento
    \item Elenco dei presenti
    \item Elenco degli assenti
    \item Redattore/autore
\end{enumerate}

Ogni verbale nel suo contenuto sarà nella forma : Questioni - Conclusioni.
Avrà infine un breve paragrafo dedito a delle considerazioni finali.



\begin{lstlisting}
main_alalal.tex
\end{lstlisting}